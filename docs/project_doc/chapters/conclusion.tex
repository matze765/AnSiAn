\chapter{Conclusion\label{ch:conclusion}}

\ac{AnSiAn} as fork of RF Analyzer has made huge structural changes to the
application architecture in order to implement a clean \ac{MVC} pattern.
Although this has improved the code quality and maintainability it also
introduced issues as described in \autoref{sec:cleanup.mem}. These issues and
their impact on other parts of the app (see \autoref{sec:morse_demod}) had to
be addressed before implementing the new features listed in
\autoref{sec:features}. There were also additional features such as support for
the SDRplay hardware or PSK31 transmission, that were added after the initial
feature list was created and therefore influenced the time schedule.

Nevertheless, all mandatory features apart from some minor exceptions
were successfully implemented. This includes new digital demodulation
modes (PSK31 and RDS), logging of demodulator outputs, support for the
rad1o badge and transmission functionality along with some basic transmission
modes (Morse and PSK31) as well as plain IQ file replay. Not implemented
was the \ac{FM}-modulated transmission of audio files and the optional features:
Walky-Talky-Mode and packet radio demodulation.

Some open tasks and improvements remain to be done in future work on the
application. These include the implementation of the complete transmission
chain as outlined in \autoref{sec:transmission}, additional modulation modes
and improvements on the \ac{BPSK} demodulation algorithm by switching to
a \ac{PLL}-based approach.
