% !TeX spellcheck = en_US
\chapter{Conclusion}
This project many focused to extend \ac{AnSiAn} for more transmission functionality using the HackRF One. Primarily, we focused on amateur radio communication modulations. We added four larger features to \ac{AnSiAn}: Firstly, we generalized the transmission chain and enabled the real-time modulation of a wide variety of signals. We validated that the transmission capabilities of the previous versions are still working and even improved, because the transmission is now starting immediately. 

Secondly, we added the \ac{RDS} transmission capability, which enables the user to operate an own radio station with the broadcast of \ac{RDS} meta data like the station name. The feature combines multiple modulations techniques that can also be useful to implement other modulation schemes. 

Thirdly, we integrated a Walkie-Talkie view into \ac{AnSiAn}, that allows a user to transmit and receive speech signals recorded from the integrated microphone. The modulation was implemented from scratch, while for the demodulation we reused modules of older versions of \ac{AnSiAn} with slight modifications. The main challenge here was to integrate the transmission chain and reception chain into one working piece that allows fast switching between reception and transmission. 

Lastly, at the end of the project we started implementing \ac{SSTV}, an amateur radio modulation scheme to transmit images using only a very low bandwidth of less than 3 kHz. The modulation and transmission of \ac{SSTV} signals was integrated into \ac{AnSiAn} which allow the user to pick an image from his device gallery and transmit it on a specified frequency.Wwe didn't manage to integrate the demodulation of \ac{SSTV} signals into AnSiAn. However, we started with a first prototype of a demodulator which still has some problems. We described the current problems and hope that it will be completed in a later version an \ac{AnSiAn}.

Having implemented all four features, we can conclude that signal processing is very cumbersome on Android, mainly because of three reasons: missing library support, computational constraints and time-consuming testing and debugging. We spent hours on finding minor problems, that would be found much easier in a traditional signal processing setup.

We are actively working with Dennis Mantz to merge back parts of this project into the original RFAnalyzer (which was the basis of AnSiAn in the first place). RFAnalyzer is a much more popular app with a lot of followers and forks. We think, that it would be great to provide the features to a large user group and that it will be received very well. 

\section{Future Work}

Although, implementing signal processing on smart phones might require more effort than implementing the same functionality with other technologies (e.g. Matlab or GNU radio), it is still worth the effort, because it is much more portable and accessible. Therefore, we believe that \ac{AnSiAn} should be continued and we propose the following features for the next versions:

Firstly, in our opinion it is required to redesign the structure of \ac{AnSiAn}, because the code base grew over time and is very complex and unmaintainable at the moment. We especially dislike the heavy use of global Preferences throughout the application, it makes the code very confusing and dependent on the Android system. Additionally, we are not sure if the use of a EventBus is a good choice here, because it is very hard to understand the control flow of the application while providing very little dependency reduction. 

Secondly, we suggest to further extend the SSTV feature. The first step would be to implement the demodulation which requires to optimize the frequency modulation. It will require quite some work, but we believe that it will be possible. After this part is done, it should be easy to add more SSTV Modes (higher resolution, RGB, ...). 

For more ideas on what to implement next, it might be a good starting point to look at what users want to be integrated into the original RFAnalyzer (see \url{https://github.com/demantz/RFAnalyzer/issues}). 


