\chapter{Project Definition\label{ch:project_definition}}

This chapter defines the mandatory and optional features that were scheduled for implementation throughout the
project and divides them into three sprints.

\section{Features\label{sec:features}}

The new features, that were to be implemented, can be divided into two groups: mandatory features of high priority within this project and optional features, that were to be implemented if time permitted. As can be seen in \autoref{sec:time_schedule}, the third sprint was reserved either for the implementation of optional features or for completing mandatory features and this documentation.

\subsection{Mandatory Features}

The following core features were scheduled for implementation during the first and
second sprint:

\begin{itemize}
	\item \ac{RDS} demodulation \\
		If the user selects the existing wide-band \ac{FM} demodulation option,
		the app shall try to detect and demodulate any existing \ac{RDS}
		signal along with the \ac{FM} audio demodulation. The extracted information
		shall be displayed on the screen.
	\item \ac{PSK31} demodulation \\
		If the user selects \ac{USB} demodulation mode, he or she shall have
		the option to enable \ac{PSK31} demodulation along with the audio
		demodulation.  The demodulated text shall appear and scroll
		through the analyzer window.
	\item Extraction of demodulated \ac{RDS}-, Morse- and \ac{PSK31}-data to logfiles \\
		If the user selects to demodulate any digital modulation, the demodulated
		text shall be written to a log file specified by the user.
	\item Support for the rad1o badge \\
		The rad1o badge, which is a modified low-cost replica of the \emph{HackRF},
		shall be supported as a signal source by \ac{AnSiAn}.
	\item Transmission support for \emph{HackRF} and rad1o \\
		If \ac{AnSiAn} is used with an \ac{SDR} capable of transmitting signals,
		it shall offer options to send signals in the following ways:
		\begin{itemize}
			\item Replay I/O samples from a file
			\item Generate and send Morse code from text
			\item FM-modulate and send audio from a file
		\end{itemize}
\end{itemize}

\subsection{Optional Features}

Optional features were scheduled for the third and last sprint. However,
they were only to be implemented if the last sprint was not needed
in order to compensate for delays on the mandatory features. The optional features are
listed in order of priority:
\begin{itemize}
	\item Walkie-Talkie Mode \\
		The user shall have the possibility to put \ac{AnSiAn} into a Walkie-
		Talkie mode. In this mode, the application will demodulate an FM channel
		and the user can quickly switch between demodulation and transmission
		of audio recorded from the internal microphone.
	\item Packet Radio demodulation\\
		A new mode \emph{Packet Radio} shall be added to \ac{AnSiAn}. Once selected, it shall allow the user
		to tune to a Packet Radio channel and display information about 
		demodulated packets on the screen. If time permits, it might even
		be possible to implement a transmission feature for Packet Radio.
\end{itemize}


\section{Time Schedule}
\label{sec:time_schedule}

The project had two developers, Dennis Mantz and Max Engelhardt,
working in three sprints. There were three milestones corresponding to
the sprints, labeled Alpha, Beta and Final Version. They each add an independent
and self-contained set of features to the application:

\begin{itemize}
	\item Software Design (due 12.05.)
	\item Sprint 1: Alpha Version (due 09.06.)
	\begin{itemize}
		\item \ac{RDS} demodulation
		\item \ac{PSK31} demodulation
		\item Extraction of \ac{RDS}-, Morse- and \ac{PSK31}-data to logfiles
	\end{itemize}
	\item Sprint 2: Beta Version (due 21.07.)
	\begin{itemize}
		\item Support for the rad1o badge
		\item Transmission support for \emph{HackRF} and rad1o
		\begin{itemize}
			\item Replay I/O samples from a file
			\item Generate and send Morse code from text
			\item FM-modulate and send audio from a file
		\end{itemize}
	\end{itemize}
	\item Sprint 3: Final Version (due 25.08.)
	\begin{itemize}
		\item Complete leftovers from previous sprints
		\item Walkie-Talkie Mode (optional)
		\item Packet Radio demodulation (optional)
	\end{itemize}
\end{itemize}
